\bab{ABSTRAK}{}
% \addcontentsline{toc}{section}{ABSTRAK}
\vspace{-1cm}

Perkembangan teknologi kendaraan otonom menuntut sistem perencanaan jalur yang mampu memahami lingkungan secara menyeluruh dan menghasilkan keputusan mengemudi yang akurat dan aman. Model perencanaan tradisional berbasis CNN seringkali memiliki keterbatasan dalam menangkap konteks global dan hubungan antar elemen lingkungan secara efektif. Untuk mengatasi hal tersebut, penelitian ini mengusulkan arsitektur SKGE-Swin yang memanfaatkan Swin Transformer dengan mekanisme skip stage guna memperkuat representasi fitur di berbagai level jaringan. Pendekatan ini memungkinkan model untuk mempertahankan informasi penting dari tahap awal hingga akhir proses ekstraksi fitur, sehingga meningkatkan kemampuan dalam memahami pola kompleks di sekitar kendaraan. Hasil eksperimen menunjukkan bahwa arsitektur SKGE-Swin mampu memberikan performa yang lebih baik dengan Driving Score sebesar 37.10 dalam tugas prediksi waypoint dibandingkan dengan metode CNN, sehingga berkontribusi pada pengembangan sistem kendaraan otonom yang lebih handal dan aman.

\vspace{0.5cm}
\textbf{Kata kunci:} multitask learning, transformer,  Kendaraan otonom, perencanaan jalur, Swin Transformer, skip stage, prediksi waypoint, representasi fitur, model deep learning.
