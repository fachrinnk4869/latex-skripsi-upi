\bab{BAB II}{TINJAUAN PUSTAKA}


% \addcontentsline{toc}{chapter}{BAB II TINJAUAN PUSTAKA}
\setcounter{chapter}{2}
\setcounter{section}{0} % RESET section ke 0
\renewcommand{\thesection}{\thechapter.\arabic{section}}
\section{Kendaraan Otonom dan Tantangan dalam Navigasi
}
Penelitian \textcite{Ni2020-oa} menggunakan sensor LIDAR, GPS, IMU untuk mengatasi permukaan jalan yang tidak rata dan mengidentifikasi rintangan di sekitar kendaraan. Dengan memanfaatkan data dari berbagai sensor, sistem kendaraan otonom dapat meningkatkan persepsi lingkungan dan kemampuan navigasi secara signifikan.
